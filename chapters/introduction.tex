\chapter{Introduction}\label{chapter:intro}
Let me introduce you to defining your own macros (Section~\ref{sec:intro:macros}) and customizing the template using preamble (Section~\ref{sec:intro:layouts}).


\section{Basic commands}

Here are some examples of the basic things.
\begin{itemize}
    \item Citing things
    \item Figures
    \item Tables
    \item Algorithm
    \item Maths
    \item Code
\end{itemize}

Cite things like \citet{adrian1979ConditionalEddiesIsotropic} and \citep{agostini2020ExplorationPredictionFluid}.

Insert a figure (like Figure~\ref{fig:example}), by using the \lstinline[language=tex]!figure! environment.
\begin{figure}[h]
    \centering
    \includegraphics[width=3in]{figs/example_image.jpg}
    \caption{
        Photo by Hans-Jurgen Mager on 
        \href{https://unsplash.com/photos/polar-bear-on-snow-covered-ground-during-daytime-qQWV91TTBrE?utm_content=creditCopyText&utm_medium=referral&utm_source=unsplash}{Unsplash}
    }\label{fig:example}
\end{figure}

Insert a table (like Table~\ref{tab:example}), by using the \lstinline[language=tex]!table! environment.
\begin{table}[h]
    \centering
    \caption{An example table}\label{tab:example}
    \begin{tabular}{|c|c|}
        \hline
        \textbf{Column 1} & \textbf{Column 2} \\
        \hline
         & something here \\
        \hline
    \end{tabular}
\end{table}

Write an algorithm with the package \emph{algorithm2e}, which we have already included in the \emph{preamble.sty}.


\begin{algorithm}
\caption{Some algorithm}\label{algo:example}
\SetKwFunction{out}{$\mat{C}$}
\SetKwInOut{Input}{Input}
\SetKwInOut{Output}{Output}

\Input{
    $\mat{A}$ - Pre-computed matrix $\mat{A}$\\ 
    $\mat{B}$ - Pre-computed matrix $\mat{B}$\\ 
    $\alpha$ - A user defined coefficient
}
\Output{
    \out - The final output of this algorithm
}

\out $\gets \mat{A} +\alpha \mat{B}$ \tcp{start with the output of the network}

\KwRet\out

\end{algorithm}

\FloatBarrier%
Figures, tables and algorithms are all floats, meaning that they may get moved to a different page as you write depending on how much space there is. 
Of course, there are other things that are also floats.
Use \lstinline[language=tex]!\FloatBarrier! to place all the floats defined before this line above this line in the text. 

Write an equation inline like this: $5\sin{\theta}$, or like 
\begin{equation}
    a = 25.
    \label{eq:equation-environment-example}
\end{equation}
Cross-reference this equation like Equation~\eqref{eq:equation-environment-example}

Write codes inline \lstinline!import numpy!.
Or write codes in a block.
Right now I have set the styles to be \lstinline{Python}, but you can change that.
\begin{lstlisting}
import numpy as np

a = np.arange(10)*0.2
a[2] = 0.0
print(a)
\end{lstlisting}

\section{Defining macros}\label{sec:intro:macros}
Define all new commands that you plan to use repeatedly in \emph{mymacros.sty}. 
Commands can be defined three different ways: \lstinline[language=tex]!\edef!, \lstinline[language=tex]!\def!, or \lstinline[language=tex]!\newcommand!.

For now I have defined the following commands:
\begin{itemize}
    \item \lstinline[language=tex]!\comment{your text}! - makes your text red.
    \item \lstinline[language=tex]!\high{superscript}! - make superscript.
\end{itemize}
And these Maths mode commands:
\begin{itemize}
    \item \lstinline[language=tex]!\mat{}! - matrix.
    \item \lstinline[language=tex]!\vector{}! - vector.
    \item \lstinline[language=tex]!\Tr{}! - trace of a matrix.
    \item \lstinline[language=tex]!\argmin! - print $\argmin$.
    \item \lstinline[language=tex]!\sym{}! - for acronyms in Maths mode.  
    \item \lstinline[language=tex]!\Rey! - print Reynolds number $\Rey$
\end{itemize}

At the top of \emph{main.tex}, we import our command file by using the line \lstinline[language=tex]!\usepackage{mymacros}!.
When writing equations, we can use the commands we defined in \emph{mymacros.sty}.
\begin{equation}
    \sym{MSE} = \| \mat{A} \|^2_2
\end{equation}
\begin{equation*}
    \Rey = 1/\nu
\end{equation*}

We can use our other defined commands as well, such as \lstinline[language=tex]!\comment! \comment{to make text red.}


\section{Changing layouts and working with preamble}\label{sec:intro:layouts}
Customizable features 
\begin{itemize}
    \item Page margin
    \item import packages
    \item font and fontsize
    \item bibliography style, and the page format
    \item title page style
\end{itemize}
