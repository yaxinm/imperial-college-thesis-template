\chapter{Introduction}\label{chapter:intro}
Let me introduce you to defining your own macros (Section~\ref{sec:intro:macros}) and customizing the template using preamble (Section~\ref{sec:intro:layouts}).


\section{Basic commands}
Cite things like \citet{adrian1979ConditionalEddiesIsotropic} and \citep{agostini2020ExplorationPredictionFluid}.

\begin{itemize}
    \item figure
    \item table
    \item algorithm
    \item math
    \item code
\end{itemize}


\section{Defining macros}\label{sec:intro:macros}
Define all new commands that you plan to use repeatedly in \emph{mymacros.sty}.

At the top of \emph{main.tex}, we import our command file by using the line \lstinline[language=tex]!\usepackage{mymacros}!.
When writing equations, we can use the commands we defined in \emph{mymacros.sty}.
\begin{equation}
    \sym{MSE} = \| \mat{A} \|^2_2
\end{equation}
\begin{equation*}
    \Re = 1/\nu
\end{equation*}

We can use our other defined commands as well, such as \lstinline[language=tex]!\comment! \comment{to make text red.}


\section{Changing layouts and working with preamble}\label{sec:intro:layouts}
Customizable features 
\begin{itemize}
    \item Page margin
    \item import packages
    \item font and fontsize
    \item bibliography style, and the page format
    \item title page style
\end{itemize}
